% Exposé for the seminar paper
% Included in main.tex

% Reflowed exposé: shorter lines and extra blank lines for readability
\section*{Exposé: Machine Learning for Prediction and Detection of Epileptic Seizures Using ECG and Other Non-Invasive Sensor Data}

\subsection*{1. Background and Motivation}

\subsubsection*{1.1 Clinical and societal relevance of epilepsy}
Epilepsy is one of the most common chronic diseases of the central nervous system,
affecting approximately 7.6 per 1,000 people \parencite{beghiEpidemiologyEpilepsy2019}.

Despite the availability of anti-seizure medications, about 30\% of patients remain drug resistant however even patients without drug resistance suffer from side effects
\parencite{kwanEarlyIdentificationRefractory2000,chenNewEraPersonalised2020}.

Unpredictable seizures lead to injuries, restrictions in daily and professional life,
fear of the next event and reduced quality of life as well as a heightened risk of accidents
\parencite{beghiAddressingBurdenEpilepsy2016,mahlerRiskInjuriesAccidents2018}.

\subsubsection*{1.2 Limitations of EEG-centric approaches}
The current gold standard for diagnosis and seizure monitoring is the electroencephalogram (EEG).
However, EEG-based approaches are often resource-intensive and tied to clinical environments.
They can be intrusive and are therefore difficult to deploy in everyday life \parencite{wongEEGDatasetsSeizure2023}.
(Wong et al., 2023)
Long-term recordings with high user comfort and minimal intrusion remain challenging.


\subsubsection*{1.3 Types of epileptic seizures}
Epileptic seizures are classified by onset as focal (starting in a localized brain region) or generalized 
(involving both hemispheres from onset). Primary generalized epilepsies commonly present with generalized tonic‑clonic seizures (GTCS),
while focal seizures can remain local or propagate to produce focal‑to‑bilateral (formerly “secondary generalized”) tonic‑clonic seizures 
(FBTCS/SGTCS). GTCS/FBTCS are associated with higher risk of injury and pronounced autonomic/cardiovascular disturbances,
which motivates monitoring peripheral cardiac signals alongside EEG 
\parencite{fisherILAEOfficialReport2017,thijsAutonomicManifestationsEpilepsy2021,beniczkyBiomarkersSeizureSeverity2020}.

\subsubsection*{1.4 Autonomic and cardiovascular changes in seizures}
Seizures commonly produce autonomic responses—most notably changes in heart rate, rhythm and HRV—that are visible in ECG/HR signals and can,
in some cases, precede clinical or electrographic onset \parencite{zijlmansHeartRateChanges2002,thijsAutonomicManifestationsEpilepsy2021}. 
Peri‑ictal tachycardia, arrhythmias and altered HRV are therefore relevant for monitoring and risk stratification,
but resting ECG alone shows limited value for reliable seizure forecasting \parencite{drakeElectrocardiographyEpilepsyPatients1993,neiEKGAbnormalitiesPartial2000}.

\subsubsection*{1.5 Wearables and non-invasive sensor modalities}
Wearable sensor technology (smartwatches, armbands, chest straps, patches) can continuously capture ECG or PPG-based HR/HRV,
accelerometry and respiration in daily life
\parencite{bonatoWearableSensorsSystems2010,beniczkyBiomarkersSeizureSeverity2020,villanuevaMultimodalMinimallyInvasive2023,wuNovelSeizureDetection2024}.

Commercially available wearables have demonstrated clear success for detecting convulsive/tonic–clonic seizures in several studies, 
with some devices reporting sensitivities often above 85–90\% and acceptable false‑alarm rates in controlled or selected cohorts.
 Multimodal systems (motion + autonomic sensors, or EMG + motion) tend to improve detection and reduce false alarms compared with unimodal approaches 
 \parencite{shumCommerciallyAvailableSeizure2021}.

At the same time, important limitations remain: only a minority of devices have peer‑reviewed, 
prospective natural‑environment evaluations or regulatory approval; most are optimized for major motor seizures and \emph{perform poorly for non‑convulsive events}; 
ambulatory use increases false alarms and depends on correct placement, smartphone connectivity and user compliance; 
and cost, comfort and caregiver availability constrain real‑world utility \parencite{shumCommerciallyAvailableSeizure2021}.

This opens the possibility of seizure detection and prediction based on non-invasive, 
ambulatory biosignals particularly for **non-convulsive seizures** \parencite{alshehriComprehensiveSurveyInternet2021}.

Against this background, the planned seminar paper systematically investigates
machine-learning (ML) approaches for seizure detection for \emph{non-convulsive seizures} and prediction that do
\emph{not} primarily rely on EEG, but on ECG and other non-invasive sensor data
(e.g. PPG, accelerometer, respiration).

\subsection*{2. Objectives and Research Questions}

\subsubsection*{2.1 Overall objective}
The overall objective of the seminar paper is to analyse the current state of the art in machine-learning (ML) 
approaches for detecting and predicting epileptic seizures through non-intrusive biosignals. 
This includes Datasets, sensor types as well as machine learning models and evaluation protocols.

The work will identify research gaps and potential directions for future projects.

The focus is on human studies from roughly the last ten years that use ECG/HR/HRV
or multimodal wearable data.

\subsubsection*{2.2 Specific research questions}
The following research questions guide the work:

\begin{enumerate}
  \item \textbf{Signals and modalities:} Which non-invasive biosignals (ECG, HRV, PPG, accelerometer, respiration, etc.)
  are used for seizure detection and prediction?
  \parencite{beniczkyBiomarkersSeizureSeverity2020,mironAutonomicBiosignalsSeizure2025,sethFeasibilityCardiacbasedSeizure2023}

  \item \textbf{Features and models:} Which feature families (time, frequency and non-linear HRV measures,
  Lorenz features, multifractal descriptors, etc.) and which ML/DL models (e.g. classical classifiers, CNN/LSTM, ensembles)
  are employed, and how interpretable are these models?
  \parencite{fujiwaraEpilepticSeizurePrediction2016,abtahiIdentificationRelevantECG2025,ghaderiAdvancesMachineLearning2025}

  \item \textbf{Datasets and evaluation:} Which datasets are used for (ML) model training and
  evaluation and which ones have the potential for real-world application?
  \parencite{andradePerformanceSeizurePrediction2024,ghaderiAdvancesMachineLearning2025}

  \item \textbf{Practical deployment:} Which technical and practical challenges arise for real-world deployment
  on wearables (e.g. energy, performance and memory constraints, robustness, user acceptance)?
  \parencite{hashashEnergyAwareDistributedEdge2021,najafiVersaSensExtendableMultimodal2024,donatiGuestEditorialUltralowPower2025,sivathambooPreferencesUserExperiences2022}

  \item \textbf{Research gaps:} Where are the main research gaps and what are the implications for future academic work,
  especially on ECG-based warning systems and multimodal wearable solutions?
  \parencite{hixsonDigitalToolsEpilepsy2020,abualrobUnlockingNewFrontiers2025}
\end{enumerate}

\subsection*{3. Theoretical and Methodological Background}

\subsubsection*{3.1 Epilepsy and autonomic dysfunction}
Epileptic seizures frequently go along with characteristic changes in heart rate and rhythm,
such as tachycardia, arrhythmias and altered HRV levels
\parencite{drakeElectrocardiographyEpilepsyPatients1993,lambertsIncreasedPrevalenceECG2015,neiEKGAbnormalitiesPartial2000}.

Several works suggest that HR and HRV changes can precede clinical seizure onset
and thus could serve as predictive markers \parencite{zijlmansHeartRateChanges2002,amengual-gualPatternsEpilepticSeizure2019,masonHeartRateVariability2024}.

\subsubsection*{3.2 Non-invasive sensor systems and wearables}
In addition to 1- or multi-lead ECG, recent studies increasingly use wearable devices
with PPG, accelerometers and respiratory channels
\parencite{beniczkyBiomarkersSeizureSeverity2020,villanuevaMultimodalMinimallyInvasive2023,wuNovelSeizureDetection2024}.

Multimodal armband or patch systems enable continuous monitoring in everyday life,
but require energy-efficient and robust algorithms, as well as reliable data transmission and security
\parencite{bonatoWearableSensorsSystems2010,forooghifarResourceAwareDistributedEpilepsy2019}.

\subsubsection*{3.3 Machine learning for detection and prediction}
Early ECG/HRV-based approaches mainly rely on statistical HRV features
and classical machine-learning models \parencite{fujiwaraEpilepticSeizurePrediction2016,lealViabilityECGFeatures2017,paveiEarlySeizureDetection2017}.

More recent work increasingly employs deep learning and explainable ML.

For example, some studies identify the most relevant ECG features using SHAP
\parencite{abtahiIdentificationRelevantECG2025}.

Other works compare model classes or provide feasibility evidence and
sensor- and feature-level separability analyses in small inpatient samples
\parencite{ghaderiAdvancesMachineLearning2025,hamlinAssessingFeasibilityDetecting2021}.

In parallel, there are targeted reviews on HRV-based prediction
and on multimodal non-EEG biosignals
\parencite{masonHeartRateVariability2024,sethFeasibilityCardiacbasedSeizure2023,mironAutonomicBiosignalsSeizure2025,pordoyEnhancedNonEEGMultimodal2025}.

\subsection*{4. Planned Structure of the Seminar Paper}

The seminar paper is planned as a structured literature review, aligned with the search and review strategy drafted in the project documents. A preliminary outline is:

\begin{enumerate}
  \item \textbf{Introduction}
    \begin{enumerate}
      \item Motivation: burden of disease, limitations of EEG-based approaches \parencite{beghiAddressingBurdenEpilepsy2016,wongEEGDatasetsSeizure2023}
      \item Aim and scope (ECG/HRV and other non-EEG sensors only)
      \item Research questions
    \end{enumerate}
  \item \textbf{Background}
    \begin{enumerate}
      \item Epilepsy, seizure types and autonomic nervous system (ANS) manifestations \parencite{thijsAutonomicManifestationsEpilepsy2021}
      \item Physiology of ECG, HRV and other peripheral biosignals in relation to seizures
      \item Wearable technologies in healthcare \parencite{bonatoWearableSensorsSystems2010,alshehriComprehensiveSurveyInternet2021,donatiGuestEditorialUltralowPower2025}
    \end{enumerate}
  \item \textbf{Problem Formulation and Evaluation Criteria}
    \begin{enumerate}
      \item Definition of detection vs. prediction, pre-ictal windows
      \item Clinically meaningful metrics (sensitivity, FPR/h, time-in-warning, PPV) \parencite{andradePerformanceSeizurePrediction2024}
    \end{enumerate}
  \item \textbf{Datasets and Study Designs}
    \begin{enumerate}
      \item Clinical ECG/HRV datasets (e.g. EPILEPSIAE, Siena, proprietary long-term recordings) \parencite{fujiwaraEpilepticSeizurePrediction2016,lealViabilityECGFeatures2017,ghaderiAdvancesMachineLearning2025}
      \item Wearable and multimodal datasets (e.g. Empatica, patient-specific armband/patch systems) \parencite{beniczkyBiomarkersSeizureSeverity2020,villanuevaMultimodalMinimallyInvasive2023,wuNovelSeizureDetection2024}
      \item Validation protocols (patient-specific vs. cross-patient, prospective, pseudo-prospective) \parencite{andradePerformanceSeizurePrediction2024}
    \end{enumerate}
  \item \textbf{Feature Engineering and Modelling}
    \begin{enumerate}
      \item HRV, Lorenz and multifractal features \parencite{fujiwaraEpilepticSeizurePrediction2016,abtahiIdentificationRelevantECG2025}
      \item Classical ML models (e.g. SVM, Random Forest, ensembles) \parencite{dongTwoLayerEnsembleMethod2022}
      \item Deep learning and explainable ML approaches \parencite{abtahiIdentificationRelevantECG2025,ghaderiAdvancesMachineLearning2025}
    \end{enumerate}
  \item \textbf{Results and Comparison of Studies}
    \begin{enumerate}
      \item Performance summary by task type (detection vs. prediction)
      \item Influence of sensor setup, features and models
      \item Usability and accuracy of everyday wearables
    \end{enumerate}
  \item \textbf{Discussion}
    \begin{enumerate}
      \item Methodological limitations (small sample sizes, data leakage, unrealistic evaluation protocols) \parencite{andradePerformanceSeizurePrediction2024,kalousiosECGbasedEpilepticSeizure2024}
      \item Technical and regulatory challenges \parencite{hashashEnergyAwareDistributedEdge2021,hixsonDigitalToolsEpilepsy2020}
      \item Implications for clinical practice and future research \parencite{abuAlrobUnlockingNewFrontiers2025}
    \end{enumerate}
  \item \textbf{Conclusion and Outlook}
\end{enumerate}

\subsection*{5. Literature Search Methodology}

\subsubsection*{5.1 Databases and search strategy}

The main databases searched are IEEE Xplore, PubMed, Scopus and Google Scholar.

Search terms combine the dimensions \enquote{seizure prediction/detection}, \enquote{ECG/HRV/heart rate},
\enquote{wearable}, \enquote{PPG/EDA/accelerometer} and explicitly exclude EEG-only studies.

\subsubsection*{5.2 Inclusion and exclusion criteria}
Inclusion criteria include, among others:

\begin{itemize}
  \item Peer-reviewed original research articles or systematic reviews (approx. 2015--2025)
  \item Use of ECG, HR/HRV or other non-invasive peripheral biosignals for seizure detection or prediction
  \item Reporting of quantitative performance metrics (e.g. sensitivity, specificity, FPR/h, AUC)
\end{itemize}

Excluded are EEG-only studies and purely conceptual papers without empirical evaluation.

\subsubsection*{5.3 Data extraction and synthesis}
Data extraction will be based on a predefined table with fields for dataset,
sensors, preprocessing, features/models, validation protocol and metrics.

The extracted information will be synthesised narratively and, where appropriate, presented in comparative tables and graphs.

\subsection*{6. Expected Contribution of the Seminar Paper}

The seminar paper will provide a consolidated overview of ML-based approaches to seizure detection
and prediction using non-invasive cardiovascular and other peripheral sensor signals.

In contrast to EEG-focused reviews, the work explicitly concentrates on ECG/HRV and wearable biosignals
and links methodological aspects (feature engineering, model choice, evaluation design)
with practical questions regarding deployment on wearables and edge devices.

More specifically, the paper aims to highlight
\begin{itemize}
  \item the maturity of current methods for everyday, real-world application scenarios,
  \item key methodological pitfalls (e.g. data leakage, unrealistic warning horizons, lack of prospective evaluation),
  \item and open research questions for future Bachelor, Master and seminar projects on ECG-based warning systems
  and multimodal wearable approaches
\end{itemize}
