\section{Introduction and Motivation}

\subsection{Clinical and societal relevance of epilepsy}
Epilepsy is one of the most common chronic diseases of the central nervous system,
affecting around 7.6 per 1,000 people \parencite{beghiEpidemiologyEpilepsy2019}.

Despite the vast range of anti-seizure medications, about 30\% of patients remain drug resistant;
However, patients responding to medication often suffer from severe side effects impacting quality of life
\parencite{kwanEarlyIdentificationRefractory2000,chenNewEraPersonalised2020}.

Unpredictable seizures lead to injuries, restrictions in daily and professional life (e.g. driving bans, need for a caretaker), 
and reduced quality of life.
\parencite{beghiAddressingBurdenEpilepsy2016,mahlerRiskInjuriesAccidents2018}.

\subsection{Limitations of EEG-centric approaches}
The standard way of diagnosis and seizure detection as well as monitoring is the electroencephalogram (EEG).
And while EEGs have been proven effective through decades of research, they require stationary equipment and expert interpretation and thus far have not been made portable, 
and while there are attempts (ear-EEG) they have yet to show an acceptable accuracy \parencite{wongEEGDatasetsSeizure2023,villanuevaMultimodalMinimallyInvasive2023}.
Effective seizure prediction and detection with minimal intrusion into the wearers daily life with acceptable performance remains difficult and will be the focus of this seminar paper.

\subsection{Wearables and non-invasive sensor modalities}
Wearable sensor technology (smartwatches, armbands, chest straps) can continuously capture ECG or PPG-based HR/HRV,
accelerometry and respiration in daily life \parencite{bonatoWearableSensorsSystems2010,beniczkyBiomarkersSeizureSeverity2020,villanuevaMultimodalMinimallyInvasive2023,wuNovelSeizureDetection2024}.

Commercial wearables are already available for detecting convulsive/tonic–clonic seizures, although only a minority of devices have peer‑reviewed, tested in a prospective natural‑environment; 
Their biggest limitation remains: most are optimized for major motor seizures and \emph{perform poorly for non‑convulsive events};
Real world tests reveal increases in false alarm rates and lower sensitivities compared to controlled clinical settings \parencite{pohWearableSeizureDetection2020}.

This indicates the current pitfalls of seizure detection and prediction based on non-invasive, 
ambulatory biosignals particularly for non-convulsive seizures \parencite{alshehriComprehensiveSurveyInternet2021}.

Aware of these limitations, the seminar paper investigates machine-learning (ML) approaches for seizure detection for \emph{non-convulsive seizures} and prediction that do \emph{not} primarily rely on EEG, but on ECG and other non-invasive sensor data (e.g. PPG, accelerometer, respiration).

\section{Theoretical and Methodological Background}

\subsection{Epilepsy, seizure types and autonomic manifestations}
Epileptic seizures are classified by onset as focal (starting in a localized brain region) or generalized
(involving both hemispheres from onset). Generalized seizures present with generalized tonic‑clonic seizures (GTCS) in 88\% of cases \parencite{keranenDistributionSeizureTypes1988},
while focal seizures often remain local and sometimes propagate to produce focal‑to‑bilateral tonic‑clonic seizures (FBTCS/SGTCS). 
GTCS/FBTCS are associated with higher risk of injury and pronounced autonomic/cardiovascular disturbances,
which motivates monitoring peripheral cardiac signals alongside EEG 
\parencite{fisherILAEOfficialReport2017,thijsAutonomicManifestationsEpilepsy2021,beniczkyBiomarkersSeizureSeverity2020}.

Seizures produce autonomic responses—most notably changes in heart rate, rhythm and HRV—that are visible in ECG/HR signals and can,
in some cases, precede clinical or electrographic onset \parencite{zijlmansHeartRateChanges2002,thijsAutonomicManifestationsEpilepsy2021}.
Peri‑ictal tachycardia, arrhythmias and altered HRV are therefore relevant for monitoring and risk stratification,
but resting ECG alone shows limited value for reliable seizure forecasting \parencite{drakeElectrocardiographyEpilepsyPatients1993,neiEKGAbnormalitiesPartial2000}.

\subsection{Non-invasive sensor systems and wearables}
In addition to single or multi-lead ECG, recent studies increasingly use wearable devices with Photoplethysmography(PPG), 
accelerometers and respiratory sensors
\parencite{beniczkyBiomarkersSeizureSeverity2020,villanuevaMultimodalMinimallyInvasive2023,wuNovelSeizureDetection2024}.

Multimodal armband or patch systems enable continuous monitoring in everyday life, but require energy-efficient and robust algorithms, as well as reliable data transmission and security \parencite{bonatoWearableSensorsSystems2010,forooghifarResourceAwareDistributedEpilepsy2019}.

\subsection{Machine learning for detection and prediction}
Early ECG/HRV-based approaches mainly rely on statistical HRV features and classical machine-learning models \parencite{fujiwaraEpilepticSeizurePrediction2016,lealViabilityECGFeatures2017,paveiEarlySeizureDetection2017}.

Recent studies utilize deep learning and explainable ML. For example, some studies identify the relevant ECG features using SHAP \parencite{abtahiIdentificationRelevantECG2025}.
Other works compare model classes or provide feasibility evidence and sensor- and feature-level separability analyses in small inpatient samples \parencite{ghaderiAdvancesMachineLearning2025,hamlinAssessingFeasibilityDetecting2021}.

Reviews based on HRV prediction and on multimodal non-EEG biosignals include \parencite{masonHeartRateVariability2024,sethFeasibilityCardiacbasedSeizure2023,mironAutonomicBiosignalsSeizure2025,pordoyEnhancedNonEEGMultimodal2025}.

\subsection{Datasets, study designs and evaluation}
Clinical ECG/HRV datasets (e.g. EPILEPSIAE, Siena, proprietary long-term recordings) and wearable/multimodal datasets (e.g. Empatica, patient-specific armband/patch systems) 
are currently the main sources for model training in the studies reviewed 
\parencite{fujiwaraEpilepticSeizurePrediction2016,beniczkyBiomarkersSeizureSeverity2020,villanuevaMultimodalMinimallyInvasive2023}.

Validation protocols vary (patient-specific vs. cross-patient, prospective, pseudo-prospective) 
and substantially affect reported performance; popular metrics include sensitivity, false‑alarm rate per hour (FPR/h), 
AUC and time‑in‑warning \parencite{andradePerformanceSeizurePrediction2024}.

Frequently cited issue include small sample sizes, data leakage, unrealistic warning horizons and lack of prospective evaluation 
\parencite{andradePerformanceSeizurePrediction2024,kalousiosECGbasedEpilepticSeizure2024}.

\section{3. Objectives and Research Questions}

\subsection{Overall objective}
The Goal is to analyse the current state of the art in machine-learning (ML) approaches for detecting and predicting epileptic seizures utilizing non-intrusive biosignals. 
The main focus will be on reasearch of the most effective  datasets, sensor types, as well as machine learning models and evaluation protocols conducted in the last decade (2015-2025). 
An additional aim is to identify research gaps and potential directions for future projects.

\subsection{Specific research questions}
The following research questions guide the work:

\begin{enumerate}
  \item \textbf{Signals and modalities:} Which non-invasive biosignals (ECG, HRV, PPG, accelerometer, respiration, etc.) are used for seizure detection and prediction? \parencite{beniczkyBiomarkersSeizureSeverity2020,mironAutonomicBiosignalsSeizure2025,sethFeasibilityCardiacbasedSeizure2023}

  \item \textbf{Features and models:} Which feature families (time, frequency and non-linear HRV measures, wavelets, etc.) are extracted
   and which ML/DL models (e.g. classical classifiers, CNN/LSTM, or deep learning models) are employed, 
   and how interpretable as well as applicable to real world scenarios are these models? \parencite{fujiwaraEpilepticSeizurePrediction2016,abtahiIdentificationRelevantECG2025,ghaderiAdvancesMachineLearning2025}

  \item \textbf{Datasets and evaluation:} Which datasets are used for model training and evaluation and which ones offer the most complete and accurate representation of real-world scenarios?
   \parencite{andradePerformanceSeizurePrediction2024,ghaderiAdvancesMachineLearning2025}

  \item \textbf{Practical deployment:} Which technical and practical challenges arise for real-world deployment on wearables
   (e.g. battery life, performance and memory constraints, user acceptance)? \parencite{hashashEnergyAwareDistributedEdge2021,najafiVersaSensExtendableMultimodal2024,donatiGuestEditorialUltralowPower2025,sivathambooPreferencesUserExperiences2022}

  \item \textbf{Research gaps:} Where are the main research gaps and what are the implications for future academic work, 
  especially on ECG-based warning systems and multimodal wearable solutions? 
  \parencite{hixsonDigitalToolsEpilepsy2020,abualrobUnlockingNewFrontiers2025}
\end{enumerate}

\section{Planned Structure of the Seminar Paper}

The seminar paper is planned as a structured literature review, planned to roughly follow this structure:

\begin{enumerate}
  \item \textbf{Introduction}
    \begin{enumerate}
      \item Motivation: burden of disease, limitations of EEG-based approaches \parencite{beghiAddressingBurdenEpilepsy2016,wongEEGDatasetsSeizure2023}
      \item Aim and scope (ECG/HRV and other non-EEG sensors only)
      \item Research questions
    \end{enumerate}
  \item \textbf{Background}
    \begin{enumerate}
      \item Epilepsy, seizure types and autonomic nervous system (ANS) interconnection \parencite{thijsAutonomicManifestationsEpilepsy2021}
      \item Physiology of ECG, HRV and other peripheral biosignal changes in relation to seizures
    \end{enumerate}
  \item \textbf{Problem Formulation and Evaluation Criteria}
    \begin{enumerate}
      \item Definition of detection vs. prediction, pre-ictal windows
      \item Clinically meaningful metrics (sensitivity, FPR/h, time-in-warning, PPV) \parencite{andradePerformanceSeizurePrediction2024}
    \end{enumerate}
  \item \textbf{Datasets and Study Designs}
    \begin{enumerate}
      \item Clinical ECG/HRV datasets (e.g. EPILEPSIAE, Siena, proprietary long-term recordings) \parencite{fujiwaraEpilepticSeizurePrediction2016,lealViabilityECGFeatures2017,ghaderiAdvancesMachineLearning2025}
      \item Wearable and multimodal datasets (e.g. Empatica, patient-specific armband/patch systems) \parencite{beniczkyBiomarkersSeizureSeverity2020,villanuevaMultimodalMinimallyInvasive2023,wuNovelSeizureDetection2024}
      \item Validation protocols (patient-specific vs. cross-patient, prospective, pseudo-prospective) \parencite{andradePerformanceSeizurePrediction2024}
    \end{enumerate}
  \item \textbf{Feature Engineering and Modelling}
    \begin{enumerate}
      \item Extracted sensor data and features \parencite{fujiwaraEpilepticSeizurePrediction2016,abtahiIdentificationRelevantECG2025}
      \item Classical ML models (e.g. SVM, Random Forest, ensembles) \parencite{dongTwoLayerEnsembleMethod2022}
      \item Deep learning and explainable ML approaches \parencite{abtahiIdentificationRelevantECG2025,ghaderiAdvancesMachineLearning2025}
    \end{enumerate}
  \item \textbf{Results and Comparison of Studies}
    \begin{enumerate}
      \item Performance summary by task type (detection vs. prediction)
      \item Influence of sensor setup, features and models
      \item Usability and accuracy of everyday wearables
    \end{enumerate}
  \item \textbf{Discussion}
    \begin{enumerate}
      \item Methodological limitations (small sample sizes, data leakage, unrealistic evaluation protocols) \parencite{andradePerformanceSeizurePrediction2024,kalousiosECGbasedEpilepticSeizure2024}
      \item Technical and regulatory challenges \parencite{hashashEnergyAwareDistributedEdge2021,hixsonDigitalToolsEpilepsy2020}
      \item Implications for clinical practice and future research \parencite{abuAlrobUnlockingNewFrontiers2025}
    \end{enumerate}
  \item \textbf{Conclusion and Outlook}
    \begin{enumerate}
    \item Summary of key findings
    \item Future directions for further research and development
    \end{enumerate}
\end{enumerate}

\section{Literature Search Methodology}

\subsection{Databases and search strategy}

Databases searched were: IEEE Xplore, PubMed, Scopus and Google Scholar.

Search terms combine the dimensions \enquote{seizure prediction/detection}, \enquote{ECG/HRV/heart rate},
\enquote{wearable}, \enquote{PPG/EDA/accelerometer} and explicitly exclude EEG-only studies.

\subsection{Inclusion and exclusion criteria}
Inclusion criteria include:

\begin{itemize}
  \item Peer-reviewed original research articles or systematic reviews (approx. 2015--2025)
  \item Use of ECG, HR/HRV or other non-invasive peripheral biosignals for seizure detection or prediction
  \item Reporting of quantitative performance metrics (e.g. sensitivity, specificity, FPR/h, AUC)
\end{itemize}

Excluded are EEG-only studies and purely conceptual papers without empirical evaluation
and studies with small sample sizes or incomparable evaluation protocols.

\subsection{Data extraction and synthesis}
Data extraction will be based on a table for included studies with fields for dataset,
sensors, preprocessing, features/models, validation protocol and metrics.

\section{Expected Contribution of the Seminar Paper}

the work will explicitly concentrate on ECG/HRV (excluding EEG) from wearable biosignals, 
and the best techniques for training and deploying effective machine learning models for seizure detection 
and prediction in real-world scenarios.

More specifically, the paper aims to highlight
\begin{itemize}
  \item the maturity of current methods for everyday, real-world application scenarios,
  \item key methodological pitfalls (e.g. data leakage, unrealistic warning horizons, lack of prospective evaluation),
  \item and open research questions for future Bachelor, Master and seminar projects on ECG-based warning systems
  and multimodal wearable approaches
\end{itemize}
