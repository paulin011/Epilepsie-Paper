\section{Introduction}
Epilepsy is one of the most common chronic diseases of the central nervous system,
affecting around 7.6 per 1,000 people \parencite{beghiEpidemiologyEpilepsy2019}.
Despite this, epilepsy remains poorly understood, resulting in 30\% of patients being drug resistant
and patients responding to the drugs report severe side effects and subsequent impacts on quality of life.
\parencite{kwanEarlyIdentificationRefractory2000,chenNewEraPersonalised2020}.
The prediction and detection of epileptic seizures can be helpful by enabling timely
drug administration, the anticipation and inhibition of seizure-related accidents as well as prompt first aid responses.

When treatment fails patients are at risk for sudden death of epilepsy (SUDEP), 
as well as physical injuries related to muscle spasms and falls during seizures.
\parencite{hesdorfferCombinedAnalysisRisk2011}.

This highlights the need for reliable and acessible seizure detection and prediction methods.
In clinical settings reliable detection and prediction has alerady been achived using EEG, however there
is a lack of solutions for ambulatory monitoring since EEG requires bulky and obtrusive equipment \parencite{chenNewEraPersonalised2020}.

The rapid introduction of wearable technology specifically regarding health monitoring,
paired with improvements in the machine learning sector has opened up the possibility of
dependable, accessible and affordable seizure detection and prediction in ambulatory settings.

This literature review aims to consolidate the current state of research regarding seizure detection and prediction
using wearable technology and machine learning methods to identify the most promising methods and modalities
, give a concise overview and and highlight future directions for research in this area.

\section{Technical Background}

\subsection{Epileptic Seizures}
Epileptic seizures are classified by their onset as focal (starting in a specific area of the brain)
or generalized (affecting both hemispheres from the onset).
\begin{enumerate}
    \item \textbf{Focal Seizures}: Seizures are further split up into focal aware and focal impaired awareness seizures,
    depending on whether the patient is aware during the seizure or not. and by onset features (motor vs non‑motor).
    Focal seizures may evolve to bilateral tonic–clonic seizures.
    \item \textbf{Generalized Seizures}: Generalized seizures begin in both cerebral hemispheres and include motor
     (e.g. tonic–clonic, myoclonic, atonic) and non‑motor (absence) types.
\end{enumerate}
\parencite{fisherInstructionManualILAE2017}.
\textbf{Note:} Tonic-clonic seizures describe a type of seizure starting with a brief tonic phase of sustained muscle contraction followed
by a clonic phase of followed by a clonic phase of rhythmic jerking; they commonly cause loss of consciousness and carry increased risk of injury and SUDEP \parencite{hesdorfferCombinedAnalysisRisk2011}.



\subsection{Wearable Technology}

Wearable technology refers to electronic devices 
(armbands, watches, chest straps) that are worn on the body,
often equipped with sensors to monitor heart rate, acceleration, 
respiration, skin conductance and blood pressure.
Detection performance of wearables for generalized tonic-clonic seizures(GTCS) is already quite high, 
due to the fact that these seizures produce large stereotyped motor patterns as well as heart-rate and EDA changes.
Several of these devices have regulatory clearances for GTCS detection 
\parencite{sethFeasibilityCardiacbasedSeizure2023,mironAutonomicBiosignalsSeizure2025}.

However focal seizures often lack large body movements and less pronounced heart-rate and EDA,
emphasizing the need for improved wearable seizure detection methods.

\subsection{Machine Learning}

Machine learning on wearable biosignals (ECG/HRV, EDA, ACC, PPG) typically frames seizure detection
and prediction as standard classification. Classical models (e.g., SVM, RF, kNN) on HRV/EDA/ACC features 
achieve solid detection and some prediction, but often in retrospective designs 
\parencite{sethFeasibilityCardiacbasedSeizure2023,masonHeartRateVariability2024}. 

Deep architectures (1D-CNN, LSTM) and multimodal fusion remain the most promising improving robustness and phase classification, 
especially when combining ECG/EDA/ACC with attention/fusion layers 
\parencite{yangMultimodalAISystem2022,pordoyEnhancedNonEEGMultimodal2025}. 

ECG-focused studies add interpretable features and out-of-distribution evaluation but emphasize non-stationarity, 
class imbalance, and scarce prospective validation as key limitations 
\parencite{abtahiIdentificationRelevantECG2025,kalousiosECGbasedEpilepticSeizure2024}.

\subsection{Patient-Specific Fine-Tuning}
Inter-patient variability in autonomic responses and preictal dynamics is substantial; 
the most informative features often differ by patient, which degrades population-trained models. 
Patient-specific fine-tuning typically improves sensitivity and lowers false alarms in ambulatory, alarm-based settings 
\parencite{andradePerformanceSeizurePrediction2024}. 

\subsection{Datasets and Synthetic Data}
Public datasets for seizure detection/prediction are largely EEG‑centric with auxiliary ECG 
(TUH, EPILEPSIAE, RPAH, CHB‑MIT), while truly wearable multimodal corpora are scarce; 
one of the few open non‑EEG sources is the Open Seizure Database (ACC+HR) 
\parencite{yangMultimodalAISystem2022,pordoyEnhancedNonEEGMultimodal2025}. 
Cross‑dataset and pseudo‑prospective evaluation across TUH→RPAH/EPILEPSIAE and across EPILEPSIAE/CHB‑MIT/AES/Epilepsy 
Ecosystem exposes domain shift and favors alarm‑based over sample‑based metrics 
\parencite{yangMultimodalAISystem2022,andradePerformanceSeizurePrediction2024}. Given rare events, imbalance, and noisy labels, studies use augmentation (time‑warping, jitter, scaling) and synthetic data to balance preictal/ictal segments, with patient‑preserving splits and external validation to avoid device/site overfitting \parencite{kalousiosECGbasedEpilepticSeizure2024,sethFeasibilityCardiacbasedSeizure2023}.

\section{Methodology}
\begin{enumerate}
    \item \textbf{Scope:} human studies in journals (2015--2025) on non-EEG wearable/autonomic biosignals (ECG/HRV, PPG, EDA, ACC) for seizure detection or prediction.
    \item \textbf{Search:} Google Scholar (scoping), IEEE Xplore (all queries), PubMed and Scopus (targeted); predefined query strings; results deduplicated.
    \item \textbf{Screening:} title/abstract screen, then full-text eligibility; backward/forward citation chasing on key papers.
    \item \textbf{Extraction:} modality; dataset/type (wearable vs clinical); dataset name/availability; task (detection vs prediction); evaluation design (retrospective/pseudo-prospective/prospective); metrics (sensitivity, FAR/h, AUC); personalization; cohort size/class balance.
    \item \textbf{Quality flags:} sample- vs alarm-based evaluation; patient-preserving splits; cross-dataset/external validation.
    \item \textbf{Synthesis:} narrative grouping by modality and task with emphasis on ambulatory feasibility and edge constraints.
\end{enumerate}

\section{Overview of literature Review}